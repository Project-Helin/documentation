\newglossaryentry{Clusteranalyse}{
	name=Clusteranalyse,
	description={Ein Verfahren zur Feststellung von Ähnlichkeiten in Datenbeständen},
	first={Clusteranalyse}
}

\newglossaryentry{AMQP}{
	name=AMQP,
	description={ Advanced Message Queuing Protocol, wird verwendet für die Übertragung der Messages an RabbitMQ }
}

\newglossaryentry{APK}{
	name=APK,
	description={ Android Package, ist das Datenformat indem Android Apps gespeichert werden. }
}

\newglossaryentry{OTG}{
	name=OTG,
	description={ USB On-the-go, ist der Standard um eine direkte Kommunikation zwischen USB-Geräten ohne Host-Controller zu ermöglichen. In unserem Fall um ein USB Gerät an ein Smartphone anzuschliessen. }
}

\newglossaryentry{Message-Producer}{
	name=Message Producer,
	description={ Bei einem Messaging-System oder einer MOM (Message Oriented Middleware) gibt es immer Message Producer und Message Consumer. Ein Message Producer erstellt Nachrichten, ein Message Consumer hingegen empfängt und verarbeitet diese. }
}


\newglossaryentry{MAVLink}{
	name=MAVLink,
	description={ MAVLink ist ein reines Header-Protokoll, dass es ermöglicht unbemannte Fahr- und Flugzeuge zu kontrollieren. Es kann über verschiedene Protokolle laufen (USB, UDP, TCP) }
}

\newglossaryentry{Flight-Controller}{
	name=Flight-Controller,
	description={ Der Flight-Controller ist der Kern der Drohne, übernimmt die Stabilisierung und Steuerung der Rotoren. Wird über das MAVLink Protokoll mit Befehlen gesteuert. }
}

\newglossaryentry{CGAL}{
	name=CGAL,
	description={The Computational Geometry Algorithms Library ist eine C++ Bibliothek die einfachen und effizienten Zugang zu Algorithmen auf geometrischen Objekten ermöglicht. Sie bietet unter anderem folgende Algorithmen: ConvexHull, AlphaShape, MeshGeneration, ShapeAnalysis, uvm.}
}

\newglossaryentry{SFCGAL}{
	name=SFCGAL,
	description={Simple Feature CGAL ist eine Bibliothek die mit Geometrie Typen aus dem OGC Standart arbeitet. Es kann somit mit den bekannten Datentypen die im OpenSource GIS Umfeld weit verbreitet sind, gearbeitet werden.}
}

\newglossaryentry{OGC}{
	name=OGC,
	description={Open Geospatial Consortium ist eine Organisation die viele GIS relevante Standarts im Opensource bereich geschaffen hat.}
}

\newglossaryentry{MIT-Lizenz}{
	name=MIT-Lizenz,
	description={Eine Software-Lizenz, die es Jedem Erlaubt den Code in jeder Art weiterzuverwenden.}
}

\newglossaryentry{BEC}{
	name=BEC,
	description={Battery Eliminator Circuit ist eine im RC-Modellbau verwendete Terminologie für eine Spannungsstabilisierungsschaltung, um konstante Spannung zu gewährleisten.}
}

\newglossaryentry{Java Topology Suite}{
	name=Java Topology Suite (JTS),
	description={Java Topology Suite ist ein API für 2D Objekte und deren Operationen. Sie erfüllt den Open Geospatial Consortium Standard und implementiert somit die gleichen Objekte wie PostGIS.}
}

\newglossaryentry{UAV}{
	name=Unpiloted Aerial Vehicle (UAV),
	description={Flugzeug welches Autonom fliegen kann, beispielsweise eine Drohne.}
}

\newglossaryentry{CRUD}{
	name= CRUD,
	description={Abkürzung für C = Create, R = Read, U = Update, D = Delete. Es wird zur Bezeichnung von Operationen auf einem Model/Resource verwendet. Beispielsweise sollen Benutzer erstellt (Create), angesehen(Read), geändert (Update) und gelöscht (Delete) werden können.}
}

\newglossaryentry{DTO}{
	name= Data Transfer Object (DTO),
	description={Objekte die übertragen werden können und keine Abhängigkeiten zum System haben.}
}

\newglossaryentry{E2E}{
	name= End-To-End Testing (E2E),
	description={Diese Art von automatisierten Test simuliert einen Benutzer. Typischerweise wird die zu testende Applikation in einem Browser geöffnet und automatisiert die gewünschten Aktionen mit Klicks ausgeführt.}
}

\newglossaryentry{Integration}{
	name= Integration Testing,
	description={Integration-Tests testen normalerweise ein ganzes System ohne Benutzeroberfläche aber mit Datenbank und anderen }
}

