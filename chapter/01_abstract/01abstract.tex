\newpage
\addcontentsline{toc}{chapter}{Abstract}
\chapter*{Abstract}

Ziel dieser Arbeit war es, einen Prototyp eines automatisierten Drohnen-Liefersystems zu konzipieren, zu entwickeln und zu testen. Um die Tests unter realen Bedingungen durchführen zu können, wurden zusätzlich zwei Multicopter zusammengestellt und gebaut.\\

Um ein System aufzubauen, dass eine Flotte von Drohnen über eine zentrale Applikation verwalten kann, wurde jede Drohne über ein Smartphone an eine Message Oriented Middleware (RabbitMQ) angebunden. Ausserdem verbindet sich die App über das \Gls{MAVLink} Protokoll mit dem \Gls{Flight-Controller} (Autopilot) der Drohne. 
Die entwickelte Android-App für das Smartphone sendet laufend Telemetrie-Daten an den Server und dient auch als Benutzer-Interface für die Beladung. \\

Der zentrale Server berechnet aufgrund der Position des Kunden eine Route, die nur über vordefinierte und sichere Flugzonen führt. Anbieter können die Flugzonen über eine Benutzeroberfläche definieren und damit statische Hindernisse umgehen. Während einer Lieferung kommuniziert der Server mit allen Clients bidirektional um sowohl Kunden, wie auch Anbietern eine Echtzeitverfolgung der Drohnen zu ermöglichen. \\

Während dieser Arbeit konnten zwei Drohnen mit einer selbst konzipierten Halterung für das Smartphone, sowie einer Abwurfvorrichtung ausgestattet werden und damit erfolgreich Lieferungen ausgeführt werden. Das mandantenfähige System steht zu Testzwecken als Software as a Service unter \url{http://my.helin.ch} zur freien Verfügung. Der Quellcode wurde unter der MIT-Lizenz auf Github veröffentlicht.
