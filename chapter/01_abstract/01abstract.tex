\newpage
\addcontentsline{toc}{chapter}{Abstract}
\chapter*{Abstract}
Ziel dieser Arbeit war es einen Prototyp eines automatisierten Drohnen-Liefersystems aufzubauen und zu zeigen, was mit heutigen Technologien in diesem Bereich möglich ist. Ausserdem beinhaltete die Arbeit die Zusammenstellung und den Aufbau von zwei Multicoptern um das System unter realen Bedingungen testen zu können.

Um ein System aufzubauen, dass eine Flotte von Drohnen über eine zentrale Applikation verwalten kann, wurde jede Drohne über ein Smartphone an eine Message Oriented Middleware (RabbitMQ) angebunden. Damit kann eine hohe Fehlertoleranz bezüglich der zu erwartenden Verbindungsabbrüche zwischen Server und Drohne erreicht werden. Ausserdem verbindet sich die App über das \Gls{MAVLink} Protokoll mit dem \Gls{Flight-Controller} (Autopilot) der Drohne. Die entwickelte Android-App für das Smartphone sendet laufend Telemetrie-Daten an den Server, dient aber auch als Benutzer-Interface für die Beladung der Drohne. 

Der zentrale Server berechnet auf Grund der Position des Kunden eine Route, die über vordefinierte und sichere Flugzonen führt. Dadurch kann Kollisionen mit statischen Objekten vorgebeugt werden. Während einer Lieferung kommuniziert der Server mit allen Clients bidirektional um sowohl Kunden, wie auch Anbietern eine Echtzeitverfolgung der Drohnen zu ermöglichen. Dies wird je nach Anforderung an die Zuverlässigkeit über das AMQP-Messaging-Protokoll oder über Websockets erreicht.  

Während dieser Arbeit konnten zwei Drohnen mit einer selbst konzipierten Halterung für das Smartphone, sowie einer Abwurfvorrichtung ausgestattet werden und damit erfolgreich Lieferungen ausgeführt werden. 



