\newpage
\addcontentsline{toc}{chapter}{Abstract}
\chapter*{Abstract}
Jeff Bezos, CEO von Amazon, gab am 2. Dezember 2013 bekannt, dass Amazon in Zukunft Pakete mit einer Drohne ausliefern wird. Das Produkt mit dem Namen Amazon Prime Air sollte im Jahr 2014 starten und Pakete bis 2.5kg ausliefern können. \\
Bis heute werden Pakete immer noch mit einem klassischen Kurrierdienst ausgeliefert. Welche Gründe machen eine Ausliereung von Gütern mittels Drohne so kompliziert? Diese zentralle Frage stand im Vordergrund der Bachelorarbeit und war der Start für Project Helin. 
Projekt Helin sollte die Möglichkeiten und Limitierungen von Drohnen und deren Servicemöglichkeit aufzeigen.
Neben konzeptionellen und wirtschaftlichen Überlegungen stand natürlich das Software Engineering im Vordergrund. Als Produkt ist eine Plattform enstanden um Drohnen als Service anzubieten. Um das Proof of Concept zu vervollständigen wurden 2 Drohnen angeschaft und aufgebaut, um abschliessende Aussagen zu den Möglichkeiten treffen zu können. Das Resultat ermöglicht es nun einem Betreiber verschiedene Dienstleistungen mit Drohnen anzubieten. Die Voraussetzungen sind nicht an ein spezifisches Model gebunden, sondern setzten blos das weitverbreitete MavLink Protokoll voraus. Die Möglichkieten sind vielseitig und die Auslieferung von Getränken oder medizinischen Güttern an eine Openair ist nur ein Beispiel. Es war von grosser Bedeutung, die gesammte Plattform  Weiter wurde ein hoher Wert auf eine offene Plattform geleget, die Möglichkeiten beschränken sich nicht nur auf Lieferdienstleistungen, weitere Möglichkeiten wurden konzeptionel offen gelassen. Sämtliche Software Bestandteile wurden OpenSource gehalten und können weiterverwendet und weiterentwickelt werden.