\chapter{Pitfalls Play mit Java Framework}
\label{ch:play_pitfalls}

\subsubsection{Dokumentation}

Viele neue Komponenten, die in Play 2.5.1 hinzukamen sind nicht ausführlich oder gar nicht Dokumentiert.

Ein Beispiel sind die WebSockets:

Bis anhin funktionierten die WebSockets über folgende einfache Schnittstelle:
\begin{lstlisting}
    return WebSocket.whenReady((in, out) -> {
        // do logic
    });
\end{lstlisting}

Dies wurde nach dem Update auf die Version 2.5.1 als deprecated markiert. Doch leider fand sich in der Dokumentation der Version 2.5.1 immer noch die veraltete Version und eine neue Dokumentation gab es nicht. Nach einer Suche im Internet fanden sich zwar Codebeispiele mit der neuen Version, allerdings benötigte die neue Version etwa 10-20 mal soviel Code und ohne viel Wissen im Bereich von AKKA-Streams hatte man keine Chance das neue Interface in einer sinnvollen Zeit zu integrieren.

\subsubsection{Community}
Normalerweise bieten solche Frameworks eine hohe Untersützung in der Community (z.B. RubyOnRails). Bei Play hingegen existieren viele Antworten, doch aufgrund der ständig änderenden APIs sind viele nicht aktuell oder es gibt sie nur für Play mit Scala. 
Es musst meist direkt bei den Entwicklern nachgefragt ( z.b. IRC Chat ) oder über Github Issues \cite{github-ticket} um an die fehlenden Informationen zu kommen.


\subsubsection{Play mit Java}
Bei allen den Negativen Punkte stellt sich die Frage, warum Play nach neun Jahren Entwicklung bei uns einen so schlechten Eindruck hinterlassen hat. 

Ein Grund könnte sein, dass Play ein Framework für zwei Sprachen (Java und Scala) zugleich ist. 