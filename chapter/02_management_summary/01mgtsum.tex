\newpage
\addcontentsline{toc}{chapter}{Management Summary}
\chapter{Management Summary}
\section*{Ausgangslage}
In den letzten Jahren haben sich die Technologien zur autonomen Steuerung von Fahr- und Flugzeugen deutlich weiterentwickelt und sind nun auch als Massenprodukte erhältlich. Im militärischen Bereich werden sie bereits länger eingesetzt, der kommerzielle Einsatz hingegen, steht erst in den Startlöchern. Eine Verkaufswelle von Flugdrohnen im Consumerbereich (Umsatz 2015 1.7 Mia.Dollar) \cite{droneNZZ} und einige Beinahezusammenstösse mit Flugzeugen, lösten in mehreren Ländern heftige Diskussionen über die Einführung von strengeren Gesetzen aus. In der kommerziellen Nutzung von autonomen Flugdrohnen zeigt sich hingegen eine gegensätzliche Entwicklung: Im Jahr 2015 wurden in den USA mindestens 2100 \cite{perm} Ausnahmegenemigungen erteilt, um autonome Drohnen zu testen. 

Unsere Vision und das Ziel der Arbeit war es deshalb, eine Plattform zu entwickeln, die es Anbietern ermöglicht, ihre Produkte mit Drohnen automatisiert an die Position des Kunden auszuliefern. Wenn also spezielle Drohnen zum autonomen Fliegen zugelassen werden, können diese mithilfe unserer Plattform auch von kleineren Anbietern verwendet werden.

\section*{Vorgehen / Technologien}

Um Drohnen über eine Internetplattform zu verwalten und steuern zu können, wurde auf jeder Drohne ein Smartphone installiert. Dieses fungiert als Vermittler zwischen der Plattform und dem Autopilot der Drohne und dient während der Beladung als Benutzeroberfläche.

Durch eine Bestellung wird eine Route berechnet, die über vordefinierte, sichere Flugzonen führt und dann einer Drohne zugewiesen. Sobald diese beladen ist, fliegt sie autonom zum Kunden, liefert das Produkt aus und fliegt wieder zurück. Während des Flugs sendet die Drohne laufend Telemetriedaten zur Plattform, welche während und nach dem Flug auf einer Karte eingesehen werden können.

\section*{Ergebnisse }

Auf Basis der von uns erarbeiteten Anforderungen haben wir eine Plattform für Anbieter, eine Android-App für die Drohne und eine Android- und iOS-App für den Kunden, konzipiert, umgesetzt und erfolgreich getestet. Für die Tests haben wir eigene Drohnen mit Standardkomponenten aufgebaut und zusätzliche Komponenten für das montieren des Smartphones und die Abwurfvorrichtung konzipiert und im 3D-Druck-Verfahren zeichnen und herstellen lassen. 

Die Plattform konnte in einer ersten Version veröffentlicht werden und steht nun bis auf weiteres als Service frei unter www.helin.ch zur Verfügung. Die Android Customer App wurde ebenfalls im Play-Store veröffentlicht. Der ganze Quellcode steht unter github.com/orgs/Project-Helin kostenlos zur Verfügung und kann für weitere Projekte verwendet werden.