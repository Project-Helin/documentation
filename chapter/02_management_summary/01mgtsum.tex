\newpage
\cleardoublepage
\phantomsection
\addcontentsline{toc}{chapter}{Management Summary}
\chapter*{Management Summary}
\section*{Ausgangslage}
Drohnen, die autonom fliegen und nicht mehr gesteuert werden müssen, gehört die Zukunft. Sie werden sicherer sein als heutige, von Menschen gesteuerte Drohnen und erlauben ein grösseres Einsatzgebiet. Firmen wie Amazon und die schweizerische Post planen bereits heute den Einsatz von Drohnen zur automatischen Auslieferung von Paketen. Doch vielen Anbietern bleibt diese Möglichkeit verwehrt, obwohl auf dem Markt eine Vielzahl von Komponenten zur Verfügung steht, um autonome Drohnen selbst zu bauen und zu betreiben. Die bestehenden Lösungen steuern aber immer nur eine Drohne gleichzeitig über Funk und bieten nur eine begrenzte Reichweite. Ausserdem existiert noch keine frei verfügbare Software, um automatisierte Dienstleistungen mit Drohnen anzubieten oder eine autonome Flotte zu verwalten.

\section*{Vorgehen / Technologien}
Um zu zeigen, was mit heutigen Technologien und Standardkomponenten bereits möglich ist, wurde ein Demonstrationssystem mit zwei selbst gebauten Drohnen entwickelt. \\

Ein Smartphone, welches auf der Drohne montiert ist, ermöglicht die Kommunikation mit der Plattform, aber auch eine Benutzerinteraktion über das Display. Bei einer eingehenden Bestellung von der Bestell-App wird automatisch eine Route zum Kunden berechnet und einer verfügbaren Drohne zugewiesen. Sobald diese beladen ist, fliegt sie autonom zum Kunden, liefert das Produkt aus und kehrt wieder zurück. Dabei bleibt die Drohne in den vordefinierten, sicheren Flugzonen und weicht somit statischen Hindernissen aus. Eine Fernsteuerung wird nicht mehr benötigt, das System ist komplett autonom. Die App dient als Schnittstelle zwischen der cloudbasierten Verwaltungssoftware und der Drohnensteuerung, welche bereits grundsätzliche Funktionen wie GPS, automatische Stabilisierung und sogar einen programmierbaren Autopiloten bietet.

\section*{Ergebnisse}
Das System zeigt erst einen Bruchteil der Möglichkeiten, die in Zukunft von autonomen Drohnen übernommen werden können. Beispielsweise können Videoaufnahmen, Infrastrukturüberwachung oder Katastrophenhilfe als Angebote integriert werden. \\

Wir sind überzeugt davon, dass die entwickelte Plattform als Denkanstoss für diese Branche und die Politik dienen kann, um die Technologien und Gesetzeslagen soweit zu verbessern, dass Dienstleistungen von Drohnen bald überall zur Verfügung stehen werden.
