\newpage
\addcontentsline{toc}{chapter}{Management Summary}
\chapter*{Management Summary}
\section*{Ausgangslage}
In den letzten Jahren haben sich die Technologien zur autonomen Steuerung von Fahr- und Flugzeugen deutlich weiterentwickelt und sind nun auch als Massenprodukte erhältlich. Im militärischen Bereich werden sie bereits länger eingesetzt, während der kommerzielle Einsatz erst in den Startlöchern steht. Eine Verkaufswelle von Flugdrohnen im Consumerbereich (Umsatz 2015 USD 1.7 Mia) \cite{droneNZZ} und einige beinahe Zusammenstösse mit Flugzeugen, lösten in mehreren Ländern heftige Diskussionen über die Einführung von strengeren Gesetzen aus. Im Gegensatz dazu, wird die Nutzung im kommerziellen Bereich immer mehr bewiligt. Im Jahr 2015 wurden in den USA mindestens 2100 \cite{perm} Ausnahmegenemigungen erteilt, um autonome Drohnen zu testen. In Zukunft kann man daher davon ausgehen, dass Drohnen für die kommerzielle Verwendung zulassungspflichtig sein werden.
\\
Unsere Vision und das Ziel der Arbeit war es deshalb, eine Plattform zu entwickeln, die es Anbietern ermöglicht, ihre Produkte mit Drohnen automatisiert an die Position des Kunden auszuliefern ohne diese auf ein spezielles Model auszulegen. Wenn also spezielle Drohnen zum autonomen Fliegen zugelassen werden, können diese mithilfe unserer Plattform auch von kleineren Anbietern verwendet werden.
\section*{Vorgehen / Technologien}
Um Drohnen über eine Internetplattform zu verwalten und steuern zu können, wurde auf jeder Drohne ein Smartphone installiert. Dieses ermöglicht die Kommunikation mit der Plattform aber auch eine Benutzerinteraktion über das Display.
\\
Bei einer eingehenden Bestellung wird eine Route berechnet und dann einer Drohne zugewiesen. Sobald diese beladen ist, fliegt sie autonom zum Kunden, liefert das Produkt aus und fliegt wieder zurück. Dabei bleibt die Drohne in den vordefinierten, sicheren Flugzonen und weicht somit statischen Hindernissen aus. Während des Fluges sendet die Drohne laufend Telemetriedaten zur Plattform, welche während und nach dem Flug auf einer Karte eingesehen werden können.

\section*{Ergebnisse }
Auf Basis der von uns erarbeiteten Anforderungen haben wir eine Plattform für Anbieter, eine Android-App für die Drohne und eine Android- und iOS-App für den Kunden, konzipiert, umgesetzt und erfolgreich getestet. Für die Tests haben wir eigene Drohnen mit Standardkomponenten aufgebaut und zusätzliche Komponenten für die Befestigung des Smartphones und die Abwurfvorrichtung konzipiert und im 3D-Druck-Verfahren herstellen lassen. 
\\
Die Plattform konnte in einer ersten Version veröffentlicht werden und steht nun bis auf weiteres als Service frei unter \url{http://www.helin.ch} zur Verfügung. Die Android Customer App wurde im Play-Store veröffentlicht. Der ganze Quellcode unterliegt der MIT-Lizenz, steht unter github kostenlos zur Verfügung und kann für weitere Projekte verwendet werden.