\newpage
\addcontentsline{toc}{chapter}{Management Summary}
\chapter*{Management Summary}
\section*{Ausgangslage}
Autonomes Fliegen ist die Zukunft und diese Autonomie wird viele Dienstleistungen ermöglichen die wir heute noch nicht kennen. Amazon und die schweizerische Post planen bereits heute den Einsatz von Drohnen zur Auslieferung von Paketen. Doch vielen kleinen Anbietern bleibt diese Möglichkeit aus Kostengründen verwehrt. Auf dem Markt existieren zur Zeit viele verschiedene Komponenten um Drohnen einfach selbst zu bauen und zu betreiben, die nötige Software zur autonomen Verwaltung der Flotte existiert aber nicht. Genau diese Plattform würde es kleinen Anbietern an OpenAir Konzerten ermöglichen Getränke mittels Drohne auszuliefern. Es wäre auch möglich Defibrilaotren im Notfall Punkt genau und schnell zustellen zu können. Diese Plattform soll kleinen Anbietern eine Chance geben Drohnen für ihr Businessmodel einzusetzten.
\section*{Vorgehen / Technologien}
Die Lösung ist ein Demonstrator der aufzeigt, dass Drohnen bereits heute verschiedene autonome Dienstleistungen anbieten können. Mithilfe zweier Apps und einer cloudbasierten Anwendungn wurde die Technologie entwickelt, um dies zu ermöglichen. Der Kunde kann mithilfe einer Xamarin App seine Bestellung tätigen. Dies geschieht geoloziert, damit die Produkte in seiner Reichweite angezeigt werden. Die Bestellung geht in unserer Cloudapplikation ein und sendet den Auftrag weiter an eine Drohne. Auf der Drohne ist ein Android fähiges Gerät installiert. Dieses übersetzt den Auftrag und die Flugroute in Koordinatenpunkte für den Autopiloten auf dem Pixhawk-Controller, welcher über MAVLink angebunden ist. Dank dem standartisierten MAVLink Protokol, ist die Plattform nicht auf einen Drohnentyp beschränkt. Es können sämtliche MAVLink fähigen Drohnen eingsetzt werden, welche heute bereits einen grossten Teil des Marktes ausmachen.

\section*{Ergebnisse}
Der Demonstrator zeigt erst einen Bruchteil der Möglichkeiten, die in Zukunft von autonomen Drohnen ausgeführt werden können. Die von uns geschaffene Ausgangslage kann weiter entwickelt werden um Remote Sensing, Drohnen Selifes oder komplexere Bestellprozesse anzubieten. Der Demonstrator verwendet ein handedlsübliches Mobiltefon, mit spezifischer Embedded Hardware, kann das Gewicht und die grösse der Drohne noch deutlich gesenkt werden, um die Transportkapzität und die Reichweite positiv zu beeinflussen. Die Möglichkeiten an diesem Punkt sind nahezu uneingeschränkt und würden helfen diese Vision salonfähig zu machen.