\newpage
\addcontentsline{toc}{chapter}{Management Summary}
\chapter*{Management Summary}
\section*{Ausgangslage}
Autonome Drohnen sind die Zukunft, denn sie werden sicherer sein als heutige, von Menschen gesteuerte Drohnen und erlauben ein grösseres Einsatzgebiet. Firmen wie Amazon und die schweizerische Post planen bereits heute den Einsatz von Drohnen zur Auslieferung von Paketen. Doch vielen Anbietern bleibt diese Möglichkeit verwehrt. Auf dem Markt existieren zur Zeit viele Komponenten um autonome Drohnen selbst zu bauen und zu betreiben. Eine Software um Dienstleistungen mit Drohnen anzubieten oder eine autonomen Flotte zu verwalten existierte bisher aber noch nicht.

\section*{Vorgehen / Technologien}
Um zu zeigen, was mit heutigen Technologien und Standardkomponenten bereits möglich ist, wurde eine Demonstrationssoftware mit zwei Drohnen konzipiert, entwickelt und getestet. \\

Ein Smartphone, dass auf der Drohne montiert ist, ermöglicht die Kommunikation mit der Plattform aber auch eine Benutzerinteraktion über das Display. Bei einer eingehenden Bestellung über das Customer App wird automatisch eine Route zum Kunden berechnet und einer Drohne zugewiesen. Sobald diese beladen ist, fliegt sie autonom zum Kunden, liefert das Produkt aus und fliegt wieder zurück. Dabei bleibt die Drohne in den vordefinierten, sicheren Flugzonen und weicht somit statischen Hindernissen aus. Eine Fernbedienung oder ähnlich wird nicht mehr benötigt, das System ist komplett autonom und wird nur über das Internet und die App gesteuert.

\section*{Ergebnisse}
Als Ergebnis ist eine Plattform entstanden, die es jedem interessierten Tüftler und Anbieter weltweit erlaubt, die aktuellen Möglichkeiten auszuloten und mit Standardkomponenten seine eigene Drohnenflotte aufzubauen um Lieferungen auszuführen.
\\
Die Plattform zeigt aber erst einen Bruchteil der Möglichkeiten, die in Zukunft von autonomen Drohnen übernommen werden können. Beispielsweise können Videoaufnahmen, Infrastrukturüberwachung oder Katastrophenhilfe als Angebote integriert werden. Ausserdem könnte das Smartphone für gewisse Aufgaben durch spezifische Embedded Hardware ersetzt werden um das Gewicht zu senken und die Transportkapazität zu erhöhen. 

Ein Smartphone, dass auf der Drohne montiert ist, ermöglicht die Kommunikation mit der Plattform aber auch eine Benutzerinteraktion über das Display. Bei einer eingehenden Bestellung wird automatisch eine Route zum Kunden berechnet und einer Drohne zugewiesen. Sobald diese beladen ist, fliegt sie autonom zum Kunden, liefert das Produkt aus und fliegt wieder zurück. Dabei bleibt die Drohne in den vordefinierten, sicheren Flugzonen und weicht somit statischen Hindernissen aus. Eine Fernbedienung oder ähnlich wird nicht mehr benötigt, das System ist komplett autonom und wird nur über das Internet und die App gesteuert.

\section*{Ergebnisse}
Das System zeigt erst einen Bruchteil der Möglichkeiten, die in Zukunft von autonomen Drohnen übernommen werden können. Beispielsweise können Videoaufnahmen, Infrastrukturüberwachung oder Katastrophenhilfe als Angebote integriert werden. Ausserdem könnte das Smartphone für gewisse Aufgaben mit spezifischer Embedded Hardware ersetzt werden um das Gewicht zu senken und die Transportkapazität zu erhöhen. \\

Als Ergebnis ist eine Plattform entstanden, die es jedem interessierten Tüftler und Anbieter weltweit erlaubt, die aktuellen Möglichkeiten auszuloten und mit Standardkomponenten seine eigene Drohnenflotte aufzubauen um Lieferungen auszuführen. Wir hoffen, dass die entwickelte Plattform als Anstoss für die Stakeholder in dieser Branche dienen kann und sich die Technologien und Gesetzeslagen insoweit verbessern, dass Dienstleistungen von Drohnen bald überall zur Verfügung stehen werden.
