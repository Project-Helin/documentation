\newpage
\addcontentsline{toc}{chapter}{Aufgabenstellung}
\chapter*{Aufgabenstellung}
\label{cha:aufgabenstellung}

\section*{Betreuer \& Auftraggeber}
\begin{itemize}
	\item{\textbf{Betreuer:} Prof Dr. Markus Stolze, Dozent für Informatik HSR}
	\item{\textbf{Co-Referent:} Prof Beat Stettler, Dozent für Informatik HSR}
	\item{\textbf{Experte:} TBD}
	\item{\textbf{Anwendungspartner:} Abteilung Informatik HSR}		
\end{itemize}

\section*{Ausgangslage}
Multicopter haben in den letzten Jahren grosse technologische Fortschritte erlebt. Sie sind leichter, einfacher zu bedienen, leistungsfähiger und preiswerter geworden, so dass sie auch für kleine Geschäfte und Privatpersonen erschwinglich geworden sind. 
Ein möglicher Anwendungsbereich von Multicoptern ist die schnelle Auslieferung von Produkten. So ist es im Prinzip schon heute möglich mit einem Multicopter einen Defibrillator punktgenau zu einer bedürftigen Person an einem Open-Air-Konzert zu bringen. Allerdings erlaubt die aktuelle Gesetzgebung keine automatisch gesteuerten Flüge von Multicoptern. Zudem ist auch manuell gesteuerte Flüge „auf Sicht“ in der Nähe von grösseren Menschenansammlungen nicht erlaubt. Da die Gesetzliche Lage aktuell im Fluss ist, kann es sein dass diese Limitationen schon bald nicht mehr gelten und es somit interessant wäre eine Software-Platform zu erstellen mit der automatisch gesteuerte Zustellung von Produkten mit Multicoptern abgewickelt werden können.  

\section*{Ziele der Arbeit}
In dieser Arbeit sollen Komponenten eine erweiterbare open-source Software-Plattform mit den folgenden vier Komponenten erstellt werden 
\begin{enumerate}
	\item{Ein Prototyp einer Android App mit der Produkt-Bestellungen beim System abgegeben werden können. Die Android App soll den Standort des Bestellers mittels GPS bestimmen und zusammen mit der Produktwahl an die zentrale Management Anwendung übertragen. Diese App soll bewusst einfach gehalten sein und soll Themen wie Bezahlung ausser Acht lassen.}
	\item{Eine prototypische Konfigurationsanwendung mit der sich Multicopter, Multicopter-Startplätze, Flugkorridore und Produktlisten erstellen und verwalten lassen.}
	\item{Eine gut getestete, wartbare und optimierte Management Anwendung welche Aufträge der Bestell-App entgegennimmt, automatisch einen freien Multicopter für die Durchführung der Zulieferung bestimmt, den Auftrag an die Multicopter-App übermittelt und die Durchführung des Auftrags überwacht. Dabei soll auch das Abbrechen eines Auftrags mitten im Flug möglich sein.}
	\item{Eine gut getestete und wartbare Multicopter App die für ein Android-Phone welches auf dem Multicopter montiert ist optimiert ist. Die Android App ist über das Handy-Netz in steter Kommunikation mit der Management Anwendung. Die App meldet die Position des Multicopters in regelmässigen Abständen und ist in der Lage im Flug Befehle der Management Anwendung an den Steuerungs-Controller auf dem Multicopter über eine für Multicopter standardisierte Schnittstelle zu übertragen.}
\end{enumerate}
Die Aufgabe umfasst neben der Erstellung der oben genannten Software-Komponenten auch die Zusammenstellung der Bestellliste für den Aufbau eines Demonstrators mit zwei flugfähigen Drohnen, sowie die Demonstration eines realistischen Bestell- und Auslieferungsszenarios. Hierzu sind zum Beispiel Lösungsvorschläge für die für Besteller gefahrlose Ablieferung zu erarbeiten. Neben der Software und der Multicopter-Hardware ist ein wichtiges Produkt dieser Arbeit eine Video-Dokumentation die zeigt inwieweit sich mit den erstellten Software und Hardwarekompoenten ein realistischen Bestell- und Auslieferungsszenarios realisieren liess.
\section*{Dokumentation}
Über diese Arbeit ist eine Dokumentation gemäss den Richtlinien der Abteilung Informatik zu verfassen. Die Dokumentation ist vollständig auf CD/DVD in einem Exemplar abzugeben (Exemplar für das Sekretariat Informatik), sowie ein Download-Link für Prof. Stolze und weitere Exemplare nach Absprache mit dem Co-Referenten (B. Stettler) und dem Experten (@@@@@).
\\
Zudem ist eine kurze Projektresultatdokumentation im öffentlichen Wiki von Prof. M. Stolze zu erstellen.
\section*{Weitere Regeln und Termine }
Im Weiteren gelten die allgemeinen Regeln zu Bachelor und Studienarbeiten \\
"Abläufe und Regelungen Studien- und Bachelorarbeiten im Studiengang Informatik" (HSR Intranet)\\ \url{https://www.hsr.ch/Ablaeufe-und-Regelungen-Studie.7479.0.html}\\
Der Terminplan ist hier ersichtlich (HSR Intranet)
\\
\url{https://www.hsr.ch/Termine-Bachelor-und-Studiena.5142.0.html}
\section*{Rechte}
Die resultierende Software und Dokumentation soll als open-source Software (MIT Lizenz) publiziert werden. Die Multicopter Hardware wird von der HSR beschafft und bleibt Eigentum der HSR. Es wird durch alle Parteien sicher gestellt, im Source-Code und im Impressum der Anwendung die originale Urheberschaft durch die HSR Studenten weiterhin sichtbar bleibt.
\\
Der Bericht der Bachelorarbeit (ohne geheime Anhänge) wird von der HSR im E-Prints Respository der HSR (eprints.hsr.ch) elektronisch veröffentlich. Titel und Abstract der Arbeit dürfen von der HSR und den Studierenden schon während der Arbeit kommuniziert werden.
\section*{Beurteilung}
Eine erfolgreiche Bachelorarbeit zählt 12 ECTS-Punkte pro Studierenden. Für 1 ECTS Punkt ist eine Arbeitsleistung von ca. 25 bis 30 Stunden budgetiert. Entsprechend sollten ca. 350h Arbeit für die Bachelorarbeit aufgewendet werden. Dies entspricht ungefähr 25h pro Woche (auf 14 Wochen) und damit ca. 3 Tage Arbeit pro Woche pro Student.
Für die Beurteilung ist der HSR-Betreuer verantwortlich. Die Bewertung der Arbeit erfolgt entsprechend der verteilten Kriterienliste.
Die Aufgabenstellung wurde am @@@@@@ vorbesprochen. Die definitive Aufgabenstellung wurde am @@@@ beschlossen. 
\begin{verbatim}


\end{verbatim}
Rapperswil, @@@@@@@ Prof. Dr. Markus Stolze \\
Institut für Software, Hochschule für Technik Rapperswil