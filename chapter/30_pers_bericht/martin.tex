Project Helin ist mir in den vergangenen Wochen sehr ans Herz gewachsen. Angefangen als eine Bachelorarbeit, hat sich daraus eine Faszination für unbemannte Flugkörper entwickelt. Der Reiz Pionierarbeit zu leisten, auf einem Gebiet, welches tagesaktuell ist, hat für mich persönlich einen komplett neuen Stellenwert bekommen. \\

Retrospektiv war diese Zeit eine sehr positive und spannende Zeit. Den wohl lehrreichsten Studienabschnitt in ein Wort zu fassen, fällt mir schwer, aber ich denke, dass 'intensiv' das Passendste ist. \\
Meine anfängliche Skepsis gegenüber Mobilgeräten, hat sich nicht nur gelegt, sondern es hat mir sogar viel Spass gemacht zu sehen wie unsere App immer benutzerfreundlicher und stabiler wurde. Das Einarbeiten in eine neue Domäne und das Entwickeln auf einer neuen Plattform, war eine Bereicherung, die ich nicht missen möchte. \\

Die Möglichkeit ein Team zu bilden, mit dem ich mehr verbinde als nur den Studienalltag, löste hohe Erwartungen in mir aus. Mit dieser erfahrenen Gemeinschaft musste es möglich sein in dieser kurzen Zeit Grosses zu erreichen. Selbstverständliche schätzte ich auch den grossen Erfahrungsschatz in diversen Bereichen. \\

Neben aller Euphorie, stelle ich selbstkritisch fest, dass dieses Semester mein persönliches Zeitmanagement etwas unter der Teilzeitbelastung gelitten hat. Für mich persönlich und meine Zukunft denke ich, dass ich mir den Konsequenzen eines solchen Projektes nun besser bewusst bin und die nötigen Stunden vorher realistischer abschätzen werde. Mithilfe dieser Erfahrung hoffe ich, dass ich etwas erwachsener und objektiver im Umgang mit einer Stundenanzahl geworden bin. Es ist kein Geheimnis, dass darunter meine persönliche Work-Life-Balance etwas gelitten hat, aber gleichzeitig einen wertvollen Lernprozess angestossen hat.
\\

Weiter so scheint es mir, war das Team ausserordentlich leistungsbereit und riss mich auch in schwierigen Arbeitsabschnitten mit. Dennoch war die Vielschichtigkeit der Aufgabenstellung durchaus ein sehr grosser Druck. Mit dem Streben nach Perfektion war das Erreichen aller Ziele in gewünschter Qualität nicht in Reichweite. Dies erforderte Kompromisse, die man durchaus bereit ist zu machen - jedoch in Anbetracht der persönlichen Ansprüche zum inneren Konflikt führt.
\\

Abschliessend kann ich sagen, dass diese Erfahrung mit zwei loyalen Freunden zu teilen, für mich ein Meilensteine des Studiums darstellt. Auf das Resultat werde ich mich, als angehender Ingenieur, noch ein Leben lang zurück erinnern.