Die Bachelorarbeit war für mich in vielerlei Hinsicht lehrreich. 
Neben dem Einsatz von neuen Technologien, war es für mich neu, die Hardware, mit der eigenen Software zu steuern. Denn bei den meisten Projekten, bei denen ich dabei war, ging es darum Business-Prozesse abzudecken. Auch mit Drohnen hatte ich vorher, ausser mit kleinen Spielzeugdrohnen, wenig Erfahrung. Insofern war die Arbeit ein Neuland für mich.\\

Nach der anfänglichen Euphorie mit der Drohne, wurde uns schnell klar, dass der grösste Teil der Arbeit das Umsetzen der Software beinhalten wird.
Ich übernahm die Verantwortung für die Serverkomponente. Dafür konnte ich meine Kenntnisse aus der Berufswelt einbringen. Mit der Wahl von Play als Framework, konnte ich auch neue Konzepte und andere Implementierungsarten ausprobieren. \\

Da zum Server auch die Routing-Komponente gehörte, hab ich einen Eindruck bekommen, was mit PostGIS möglich ist. Nun kann ich bewerten, ob sich der Einsatz einer geobasierten Datenbank in einem Projekt lohnt, und welche Möglichkeiten dadurch entstehen.\\

Als letzte grosse Komponente war ich für die Erweiterung des Customer-Apps zuständig. Dies war für mich die lehrreichste Zeit der ganzen Arbeit, denn neben dem Einsatz von C-Sharp, konnte ich mit Xamarin Forms eine App für Android und iOS entwickeln. Da konnte ich gleich auch die Unterschiede zu Java erkennen und Inspirationen holen, wie manche Probleme zu lösen sind.
Auch Konzepte, denen ich skeptisch gegenüberstand, wie etwa \textit{async/await}, erwiesen sich als sehr nützlich. Ein Erkenntnis, welche ich sicherlich mitnehme ist, dass man sich den Einsatz von Xamarin für die Crossplattform-Entwicklung bei einem nächsten Kundenprojekt überlegen könnte.\\

Das Arbeiten in einem Dreierteam war eine Herausforderung. Anfänglich waren hitzige Diskussionen an der Tagesordnung, doch mit der Zeit wurden es weniger. Das Problem bestand sicher darin, dass wir alle nebenbei Teilzeit arbeiteten und somit bereits unterschiedliche Erfahrungen und Meinungen mitbrachten.  
\\

Nachträglich gesehen war das Projekt vom Umfang her ziemlich gross. Denn auch wenn es zum Beispiel nur ein kleines Customer-App zu entwicklen gilt, muss doch die Zeit dafür eingeplant werden. Das Problem bestand aber darin, dass wir nicht auf einen der Komponente verzichten konnten. Wir wollten auch ein erfolgreiches Projekt abliefern, dazu gehört halt dass der Customer-App in einem entsprechenden Zustand ist.
\\

Als die einzelnen Komponenten anfingen zusammenzuarbeiten, und der Spass mit dem Testen der Drohne richtig anfing, wurde das Wetter auch gleich schlecht und der Code-Freeze des Projekts stand auch kurz bevor. Ich denke, dass wir eine gute Grundlage für Folgearbeiten geschaffen haben.Ich bin mir sicher, dass ich das Projekt nach dem Abschluss anschauen oder weiter ergänzen werde.
Denn die Bachelorarbeit hat mir weitere Ideen gegeben, so etwa, ob nicht ein autonomes Modellfahrzeug angebunden werden könnte. Die Bestellliste für die Teile dafür ist bereits zusammengestellt.




