Die Bacheolorarbeit war für mich in vielerlei Hinsicht lehrreich. 
Neben dem Einsatz von neuen Technologien, war es für mich neu eine Hardware, mit der Software die wir geschrieben haben zu steuern.
Denn die meisten Projekten, bei denen ich dabei war, ging es darum Business-Prozesse abzudecken.
\\

Auch mit Drohnen hatte ich vorher, ausser mit kleinen Spielzeugdrohnen, wenig Erfahrung. Insofern war die Arbeit eine Neuland für mich.
\\

Nach der anfänglichen euphorie mit der Drohne, wurde uns schnell klar, dass wir eine Serverkomponente brauchten, so übernahm ich de Verantwortung für die Serverkomponente. 
Da konnte ich auch gleich meine Kenntnisse aus der Berufswelt mit einbringen. 
Mit dem Wahl von Play als Framework, konnte ich auch neue Konzepte und andere Implementierung, derselben Anforderung wie in der Geschäftswelt, sehen. 
\\

Das 
\\

Für das Customer-App wurde für Xamarin entschieden.
Marcel Amsler hatte das Projekt aufgesetzt und die groben Masken erstellt. Ich übernahm die Anbindung an den Server.
Für mich war dies die lehrreichste Zeit der ganzen Arbeiten, denn neben dem Einsatz von C-Sharp, konnte ich gleich auch Unterschiede zu Java erkennen und auch Inspirationen holen, wie manche Probleme zu lösen sind. Auch Konzepte, denen ich skeptisch dastand, wie etwa \textit{asnych/await} erwiesen sich als sehr nützlich. Die Erkenntnis, welche ich sicherlich herausnehem ist, dass nach dem Einsatz von Xamarin für die Crossplattform Entwicklung kann ich mir gut vorstellen, diese für einen Kundenprojekt einzusetzten. 
\\

Als die einzelnen Komponenten anfangen zusammenzuarbeiten, und der Spass mit dem Testen der Drohne richtig anfing, wurde das Wetter auch gleich schlecht und der Code-Freeze des Projekts in Absicht. Ich denke, dass wir eine gute Grundlage für Folgearbeiten geschaffen haben. Denn, da nun mit der Drohne an eine Position geliefert werden, könnte der Prozess weiter optimiert und ergänzt werden. \\

Ich bin mir sicher, dass ich das Projekt nach dem Abschluss anschauen oder weiter ergänzen werde. Denn die Bachelorarbeit hat mir weitere Idee gegeben, so etwa, ob nicht ein autonomes Modellfahrzeug angebunden werden könnte. Eine Bestellliste für die Teile ist bereits zusammengestellt.











