Die Bacheolorarbeit war für mich in vielerlei Hinsicht lehrreich. 
Neben dem Einsatz von neuen Technologien, war es für mich neu eine Hardware, mit der Software die wir geschrieben haben, zu steuern.
Denn die meisten Projekten, bei denen ich dabei war, ging es darum Business-Prozesse abzudecken.

Auch mit Drohnen hatte ich vorher, ausser mit kleinen Spielzeugdrohnen, wenig Erfahrung. Insofern war die Arbeit ein Neuland für mich.
\\

Nach der anfänglichen Euphorie mit der Drohne, wurde uns schnell klar, dass wir eine Serverkomponente brauchten. Da konnte ich auch gleich meine Kenntnisse aus der Berufswelt mit einbringen. 
Mit dem Wahl von Play als Framework, konnte ich auch neue Konzepte und andere Implementierung, derselben Anforderung wie in der Geschäftswelt, sehen. 
\\

Da zum Server auch die Routing-Komponente gehörte, hab ich einen Eindruck bekommen, was mit PostGIS machen lässt. Nun kann bewerten, ob der Einsatz von GIS in einen andere Projekte sich lohnt, und welche Möglichkeiten es gibt. 
\\
Als Letzte grosse Komponente, war ich für die Erweiterung des Customer-Apps zuständig. Dies war für mich die lehrreichste Zeit der ganzen Arbeiten, denn neben dem Einsatz von C-Sharp, konnte ich auch Xamarin Forms eine App für Android und iOS entwickeln. Konnte ich gleich auch Unterschiede zu Java erkennen und auch Inspirationen holen, wie manche Probleme zu lösen sind. Auch Konzepte, denen ich skeptisch gegenüberstand, wie etwa \textit{asnych/await} erwiesen sich als sehr nützlich. Ein Erkenntnis, welche ich sicherlich herausnehme ist, dass nach dem Einsatz von Xamarin für die Crossplattform-Entwicklung kann ich mir gut vorstellen, Xamarin für ein Kundenprojekt einzusetzen. 
\\

Das Arbeiten in einem Dreierteam stellte sich als herausfordernd heraus. Anfänglich waren viele hitzigen Diskussionen üblich, doch mit der Zeit wurden es weniger, doch gab es immer wieder emotionale Diskussionen. Das Problem bestand darin, dass wir keine 'Projektleiter' hatten, wie ich dies aus einem Geschäftsumfeld kenne. Auch wenn Entscheidung im Team gefällt wurden, fehlte jemand, der diese Rolle explizit übernahm. Beim nächsten Mal hätte ich dies anders gemacht.
\\
 
Als die einzelnen Komponenten anfangen zusammenzuarbeiten, und der Spass mit dem Testen der Drohne richtig anfing, wurde das Wetter auch gleich schlecht und der Code-Freeze des Projekts in Absicht. Ich denke, dass wir eine gute Grundlage für Folgearbeiten geschaffen haben. Denn, da nun mit der Drohne an eine Position geliefert werden, könnte der Prozess weiter optimiert und ergänzt werden. 
\\

Ich bin mir sicher, dass ich das Projekt nach dem Abschluss anschauen oder weiter ergänzen werde. Denn die Bachelorarbeit hat mir weitere Idee gegeben, so etwa, ob nicht ein autonomes Modellfahrzeug angebunden werden könnte. Eine Bestellliste für die Teile ist bereits zusammengestellt.




