Die Vorfreude auf meine Bachelorarbeit war riesig. Ein spannendes Projekt, ein gutes Team und zwischendurch Draussen mit den Quadcoptern rumfliegen. Im Vergleich zur Arbeit im Geschäft scheint das fast wie Freizeit.\\

Schon in den ersten Wochen wurde uns klar, wie viel Arbeit uns bevorstand. Doch wir konnten den Scope nicht einschränken, da das Ergebnis sonst praktisch wertlos gewesen wäre. Das führte dazu, dass wir ziemlich wenig Zeit hatten, um Draussen mit den Drohnen Testflüge zu machen und dies auf das absolute Minimum beschränken mussten. Die Entwicklung machte dann in einzelnen Bereichen zwar ziemlich Fortschritte, wir verloren bei den Problemen mit dem Play-Framework aber viel Zeit. Mit dem steigenden Druck nahm der Spass am Projekt stetig ab. Was uns half waren allerdings die saubere Organisation der offenen Tasks und die stetige Anpassung der Schätzungen, dadurch konnten wir immer wieder sehen, dass doch genug Zeit war um noch alles zu schaffen.\\

Wir konnten erst spät im Projekt das ganze System testen, es machte dann aber richtig Spass wie die Drohne von ganz alleine Hindernisse umflog und den Fallschirm mit dem Getränk abwarf. Das war ein richtiges Erfolgserlebnis.\\

Wenn ich nun am Ende des Projekts über meine Verbesserungsmöglichkeiten reflektiere, muss ich sagen, dass die grössten Probleme durch Team-interne Streitigkeiten und den zu kurz gekommenen Spass entstanden sind. Mit beidem hätte ich besser umgehen können, indem ich gelassener gewesen wäre. Ich muss wohl noch besser lernen mich weniger über die Dinge aufzuregen und stattdessen eine Lösung zu suchen oder sie für den Moment ruhen zu lassen. Dies wäre mir bestimmt leichter gefallen, wenn die Vorfreude und die damit verbundenen Erwartungen nicht so gross gewesen wären. Aber gerade in solchen Projekten wäre es eigentlich noch wichtiger die persönlichen Interessen hinten anzustellen und eine gewisse Gelassenheit an den Tag zu legen. \\

Ich finde es schade, dass das Projekt schon vorbei ist. In den letzten Wochen sind so viele neue und spannende Ideen entstanden, die man jetzt mit relativ wenig Aufwand ausprobieren und entwickeln könnte.