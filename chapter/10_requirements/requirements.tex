\chapter{Anforderungen}

\section{Benutzer und Personas}


Die Benutzer der Project Helin Plattform teilen sich in zwei Gruppen auf, welche in dieser Arbeit berücksichtigt werden.

\subsection{Service-Nutzer}

Service-Nutzer verwenden nur die Service App um einen Service oder ein Gut an ihre Position zu bestellen.

\subsubsection{Persona Diego}

\begin{figure}[h]
\includegraphics[width=.35\textwidth]{images/persona-diego.jpg}
\caption{Persona Diego: Freie Lizenz; Quelle }
	%\url{http://blog.placeit.net/free-avatar-pack/}}
\label{fig:diego}
\end{figure}

Er ist \textbf{23 Jahre alt} wohnt in einer WG in Uster.\\

Diego hat eine Informatik-Lehre mit BMS abgeschlossen und ist auf der Suche nach einer neuen beruflichen oder schulischen Herausforderung.

\paragraph{Technisches Verhalten}

Arbeitet täglich acht Stunden mit dem PC. Nutzt gerne neue Technologien. Besitzt ein Android-Smartphone der neusten Generation. Die Android Updates macht er immer sofort. Interessiert sich für neue Technologien und sieht sich regelmässig Kickstarter Projekte an.\\

\paragraph{Ziele}

Möchte sich weiterbilden und neue Herausforderungen finden. Möchte neue Technologien entdecken und einsetzen.\\

\subsection{Administratoren}

Administratoren verwenden die Project Helin Plattform um eine Flotte von Drohnen zu verwalten. Sie nutzen dazu die Webseite der Plattform und kümmern sich um das definieren von Flugzonen, Services und Gütern.

\subsubsection{Persona Stefanie}
\begin{figure}[h]
	\includegraphics[width=.35\textwidth]{images/persona-stefanie.jpg}
	\caption{Persona Stefanie: Freie Lizenz; Quelle} %\url{http://blog.placeit.net/free-avatar-pack/}}
	\label{fig:stefanie}
\end{figure}


Carmen ist \textbf{35 Jahre alt} und lebt alleine in Zürich.

Sie unternimmt viel mit Freunden und reist gerne. Sie ist auf dem Land aufgewachsen und wohnt jetzt in Zürich.

Carmen arbeitet bei einer Eventagentur und hat eine Ausbildung als Eventmanagerin abgeschlossen.

\paragraph{Technisches Verhalten}
Sie nutzt den PC täglich bei der Arbeit, vor allem Planungstools und das E-Mail-Programm. Ausserdem ist sie zu Recherchezwecken viel im Internet unterwegs. Sie nutzt Chrome als Browser im Geschäft. Sie besitzt ein Samsung Galaxy S4, dass sie schon seit einigen Jahren verwendet. Zuhause hat sie keinen PC.

\paragraph{Kommunikationsverhalten}
Sie kommuniziert geschäftlich hauptsächlich per E-Mail. Mit Freunden kommuniziert sie über WhattsApp.

\paragraph{Ziele}
Sie möchte mit neuen und innovativen Ideen Events für Besucher spannender gestalten.

\section{Funktionale Anforderungen}

Die funktionalen Anforderungen leiten sich aus der Aufgabenstellung, sowie den mit dem Betreuer diskutierten Ideen ab.

\subsection{Administrator}
\begin{itemize}
\item Als Administrator möchte ich auf der Webseite einen Account erstellen können.
\item Als Administrator möchte ich ein Projekt erfassen können.
\item Als Administrator möchte ich auf eine Drohne dem Projekt hinzufügen können.
\item Als Administrator möchte ich alle Drohnen verwalten können (RUD).
\item Als Administrator möchte ich Güter verwalten können(CRUD).
\item Als Administrator möchte ich Services verwalten können (CRUD) (optional).
\item Als Administrator möchte ich Flug-, Lade- und Abwurfzonen verwalten können.
\item Als Administrator möchte ich Bestellungen verwalten können(CRUD).
\item Als Administrator möchte ich Missionen vor, nach, und während der Ausführung ansehen können.
\item Als Administrator möchte ich Missionen abbrechen können.
\end{itemize}

\subsection{Kunde}
\begin{itemize}
\item Als Service-Nutzer möchte ich eine App aus dem Google Play Store herunterladen können.
\item Als Service-Nutzer möchte ich die App nutzen ohne mich anzumelden zu müssen.
\item Als Service-Nutzer möchte ich in der App, aus einer Auswahl von Gütern und Services, eine Auswahl treffen können.
\item Als Service-Nutzer möchte ich eine Bestellung tätigen können.
\item Als Service-Nutzer möchte ich auf der Karte des Smartphones die Bewegung der Drohne verfolgen können.
\item Als Service-Nutzer möchte ich textuelle Updates über den Status meiner Bestellung erhalten.
\item Als Service-Nutzer möchte ich eine Bestellung stornieren können.
\end{itemize}

\subsection{Drone-Operator}
\begin{itemize}
	\item Als Drone-Operator möchte ich eine Android App mit Hilfe der heruntergeladenen APK installieren können.
	\item Als Drone-Operator möchte ich das Gerät bzw. die Drohne beim Server registrieren können.
	\item Als Drone-Operator möchte ich das Gerät bzw. die Drohne zu einer Organisation hinzufügen können.
	\item Als Drone-Operator möchte ich das OTG-Fähige Smartphone an einen MAV-Link kompatiblen Flight-Controller über USB anschliessen können.
	\item Als Drone-Operator möchte ich eine Verbindung zwischen App und Server über das Internet herstellen könnnen.
	\item Als Drone-Operator möchte ich eine Verbindung zwischen App und Flight-Controller herstellen können.
	\item Als Drone-Operator möchte ich den aktuellen Zustand der Verbindungen zur Drohne und zum Server sehen.
	\item Als Drone-Operator mpchte ich den aktuellen Status des Flight-Controllers, beispielsweise GPS und Batteriespannung sehen.
	\item Als Drone-Operator sehe wenn der Drohne eine neue Mission zugeteilt wurde.
	\item Als Drone-Operator möchte ich eine Liste von Gütern anzeigen lassen, die für die aktuelle Mission geladen werden müssen.
	\item Als Drone-Operator möchte ich eine Mission annehmen oder ablehnen können.
	\item Als Drone-Operator möchte ich die Beladung einer Drohne bestätigen können.
	\item Als Drone-Operator erhalte ich ein visuelles und akustisches Countdown-Signal bevor die Drohne startet.
	\item Als Drone-Operator möchte ich den Start der Drohne während des Countdowns verhindern können.
\end{itemize}


\section{Nichtfunktionale Anforderungen}

\subsection{Verbindungsabbruch}

\begin{tabular}{|p{.25\textwidth}|p{.75\textwidth}|} \hline
	Synopsis & Verbindungsabbruch der Onboard App zum Server soll keine negativen Auswirkungen auf die Mission haben.  \\ \hline
		
	Messbarkeit & Nach dem Start der Mission muss die Drohne auch ohne Verbindung zum Server die Mission abschliessen können. \\ \hline
\end{tabular}

\subsection{Sicherheit im Mittelpunkt}
\begin{tabular}{|p{.25\textwidth}|p{.75\textwidth}|} \hline
	Synopsis & Eine Drohne führt vor dem Freigeben der Motoren (Arming) einen Check durch, der prüft ob alle nötigen Voraussetzungen für einen Start erfüllt sind. Ausserdem müssen vor einem Start ebenfalls Voraussetzungen der Mission erfüllt sein, beispielsweise muss der Drone-Operator den Start freigeben. Sollte eine dieser Vorraussetzungen nicht erfüllt sein, darf die Drohne nicht starten.  \\ \hline
	
	Messbarkeit & Drohne darf nicht starten, falls der Pre-Flight-Check des Autopiloten nicht erfolgreich war oder Voraussetzungen für die aktuelle Mission nicht erfüllt sind. \\ \hline
\end{tabular}

\subsection{Verbindungswiederherstellung}
\begin{tabular}{|p{.25\textwidth}|p{.75\textwidth}|} \hline
	Synopsis & Nach einem Verbindungsabbruch zwischen dem Server und der Onboard App soll die Verbindung wiederhergestellt werden sobald wieder Internet verfügbar ist. \\ \hline
	
	Messbarkeit & Die Verbindung zwischen Server und App wird nach dem deaktivieren und wieder aktivieren der Internetverbindung innert 30 Sekunden wiederhergestellt.\\ \hline
\end{tabular}


\subsection{Nutzlast der Drohne}
\begin{tabular}{|p{.25\textwidth}|p{.75\textwidth}|} \hline
	Synopsis & Eine 1/2 Liter PET Flasche soll innerhalb von einem 500m Radius geliefert werden können. \\ \hline
	Messbarkeit & Im App kann ein Produkt mit einem Gewicht von 500g bestellt, und geliefert werden. \\ \hline
\end{tabular}


\section{Domain-Model}

Aus den Funktionalen Anforderungen ergibt sich das folgende Domainmodel.

\begin{figure}[h]
	\includegraphics[width=1.0\textwidth]{images/domainmodell.png}
	\caption{Domain-Model}
	\label{fig:domain-model}
\end{figure}

