\newpage
\chapter{Systemtest}

Die nachfolgenden Tests wurden mit den folgenden Komponenten durchgeführt:

\begin{itemize}
	\item Server befindet sich auf www.helin.ch
	\item On-Board-App: läuft auf einem Nexus 4 mit Android 4.4
	\item Customer-App: läuft auf einem Nexus 5 mit Android 6.1
	\item Customer-App wurde mit einem OTG Kabel mit der Drohne verbunden
\end{itemize}

Es wurde jeweils nach dem Implementierungs-Meilensteinen ein kompletter Test ausgeführt.

\section{Ende Proof Of Concept}

Die Tests wurden am 25.03.2016 durchgeführt.

\begin{todolist}
	\item[\done] Der Administrator kann sich erfolgreich registrieren.
	\item Es kann kein Account mit demselben Namen erstellt werden
	\item[\done] Administrator kann sich mit seinem Email und Passwort einloggen
	\item[\done] Administrator kann eine neue Organisation erstellen
	\item[\done] Administrator kann den Namen der Organisation ändern
	\item Administrator kann einen zusätzlichen Administrator zur Organisation hinzufügen
	\item Es kann ein Administrator aus der Organisation gelöscht werden
	\item[\done] Administrator kann ein neues Produkt erfassen
	\item[\done] Administrator kann ein vorhandenes Produkt editieren
	\item[\done] Administrator kann ein Produkt löschen
	\item[\done] Administrator kann ein neues Projekt erfassen
	\item Administrator kann eine Bestell-, Abwurfs-, Flug- und Ladezone definieren
	\item Administrator kann eine Zone anpassen und löschen
	\item Administrator kann ein Produkt zu einem Projekt zuweisen
	\item Administrator kann ein Produkt aus einem Projekt herauslöschen
	\item Administrator kann eine Flugroute mit einem Kunden simulieren.
	
	\item Administrator kann die Bestellung ansehen.
	\item Administrator kann die vorgeschlagene und geflogene Flugroute anschauen.
	
	\item[\done] Eine registrierte Drohne kann einem Projekt zugewiesen werden
	\item Eine registrierte Drohne kann angepasst werden
	\item Eine registrierte Drohne kann als inaktiv markiert werden
	\item Eine aktuell verbundene Drohne zeigt den letzten Akkustand an
	\item Administrator kann alle Bestellungen ansehen
	\item Administrator kann die Missionen einer Bestellung ansehen
	
	\item Administrator kann eine Bestellung ändern 
	\item Administrator kann eine Bestellung löschen 
	\item Drohne kann herausgelöscht werden 
	
	\item Administrator kann eine laufende Mission abbrechen 
\end{todolist}

\subsubsection{Nicht-Funktionale Anforderungen}
\begin{todolist}
	\item Auf die Seite kann nur über HTTPS zugegriffen werden
\end{todolist}

\subsection{Drone-Operator}
\subsubsection{Funktionale Anforderungen}
\begin{todolist}
	\item Das Onboard-App kann über ein APK heruntergeladen und installiert werden.
	\item[\done] Mit dem Onboard-App kann ich die Drohne beim Server registrieren und einer Organisation hinzufügen
	\item[\done] Das Onboard-App kann sich mit dem Server verbinden
	\item Das Onboard-App kann auf Wunsch die Verbindung zum Server trennen
	\item Die Drohne kann auf dem Onboard-App als inaktiv markiert werden
	\item[\done] Das Onboard-App kann sich mit der Drohne über USB Kabel verbinden
	\item[\done] Das Onboard-App kann die Verbindung mit der Drohne trennen
	\item Es kann der aktuelle Status des GPS, der Batterie und der Verbindung zum Server angezeigt werden
	\item Eine vorgeschlagene Mission kann abgelehnt werden
	\item Drone-Operator kann eine Mission annehmen und sieht die zu belandende Produkte
	\item Bevor die Drohne startet erhalte ich einen Countdown
	\item Bevor die Drohne startet erhalte ich einen akustisches Signal
	\item Während des Countdowns kann der Start abgebrochen werden
	\item Die Drohne kann mit den Produkten beladen und bestätigt werden
\end{todolist}

\subsubsection{Nicht-Funktionale Anforderungen}
\begin{todolist}
	\item Die Verbindung zum Server beim Registrieren ist verschlüsselt ( Server kann nur HTTPS angesprochen werden )
	\item Die Verbindung zum Server bei der Übertragung der Drohneninformation ist verschlüsselt (überprüft mittels RabbitMQ Management Konsole)
	\item[\done] Beim Verbindungsabbruch zum Server wird die Mission weitergeführt
	\item[\done] Nach einem Verbindungsabbruch wird die Verbindung automatisch wiederhergestellt
\end{todolist}

\subsection{Customer}
\subsubsection{Funktionale Anforderungen}
\begin{todolist}
	\item Der Kunde kann sich die App aus dem Google Play Store herunterladen 
	\item Kunde kann sich die bestellbaren Produkte anschauen, ohne sich einzuloggen
	\item Kunde sieht nur nur Produkte innerhalb der Bestellzone
	\item Kunde kann eine Bestellung abschicken und sieht die voraussichtliche Lieferposition
	\item Kunde kann die Bestellung abbrechen
	\item Kunde kann sich über Google Anmelden und sieht welche Berechtigung von Google erforderlich sind
	\item Kunde kann die bestellte Ware mit PayPal bezahlen ( nur im Testmodus )
	\item Kunde kann die Bestellung bestätigen
	\item Kunde kann seine Bestellungen anschauen und dessen Status ansehen
	\item Kunde kann die einzelnen Lieferungen anschauen und dessen Status ansehen
	\item Kunde kann bei einer aktuellen Lieferung die Position der Drohne mitverfolgen
	\item Als Kunde kann ich  eine Bestellung stornieren
\end{todolist}

\subsubsection{Nicht-Funktionale Anforderungen}
\begin{todolist}
	\item Die Verbindung zum Server ist verschlüsselt
	\item Ohne Internetverbindung wird beim Abruf der Produkte eine Fehlermeldung angzeigt
	\item Ohne GPS Verbindung wird beim Abruf der Produkte eine Fehlermeldung angzeigt
\end{todolist}



\section{Ende Implementierung}

Dies folgenden Tests wurden, nach dem Abschluss der Implementierung durchgeführt, am 06.06.2016 durchgeführt.

\subsection{Administrations Seite}
\subsubsection{Funktionale Anforderungen}

\begin{todolist}
	\item[\done] Der Administrator kann sich erfolgreich registrieren.
	\item[\done] Es kann kein Account mit demselben Namen erstellt werden
	\item[\done] Administrator kann sich mit seinem Email und Passwort einloggen
	\item[\done] Administrator kann eine neue Organisation erstellen
	\item[\done] Administrator kann den Namen der Organisation ändern
	\item[\done] Administrator kann einen zusätzlichen Administrator zur Organisation hinzufügen
	\item[\done] Es kann ein Administrator aus der Organisation gelöscht werden
	\item[\done] Administrator kann ein neues Produkt erfassen
	\item[\done] Administrator kann ein vorhandenes Produkt editieren
	\item[\done] Administrator kann ein Produkt löschen
	\item[\done] Administrator kann ein neues Projekt erfassen
	\item[\done] Administrator kann eine Bestell-, Abwurfs-, Flug- und Ladezone definieren
	\item[\done] Administrator kann eine Zone anpassen und löschen
	\item[\done] Administrator kann ein Produkt zu einem Projekt zuweisen
	\item[\done] Administrator kann ein Produkt aus einem Projekt herauslöschen
	\item[\done] Administrator kann eine Flugroute mit einem Kunden simulieren.

	\item[\done] Administrator kann die Bestellung ansehen.
	\item[\done] Administrator kann die vorgeschlagene und geflogene Flugroute anschauen.

	\item[\done] Eine registrierte Drohne kann einem Projekt zugewiesen werden
	\item[\done] Eine registrierte Drohne kann angepasst werden
	\item[\done] Eine registrierte Drohne kann als inaktiv markiert werden
	\item[\done] Eine aktuell verbundene Drohne zeigt den letzten Akkustand an.
	\item[\done] Administrator kann alle Bestellungen ansehen
	\item[\done] Administrator kann die Missionen einer Bestellung ansehen

	\item Administrator kann eine Bestellung ändern 
	\item Administrator kann eine Bestellung löschen 
	\item Drohne kann gelöscht werden 


\end{todolist}

\subsubsection{Nicht-Funktionale Anforderungen}
\begin{todolist}
	\item[\done] Auf die Seite kann nur über HTTPS zugegriffen werden
\end{todolist}


\subsection{Drone-Operator}
\subsubsection{Funktionale Anforderungen}
\begin{todolist}
	\item[\done] Das Onboard-App kann über ein APK heruntergeladen und installiert werden.
	\item[\done] Mit dem Onboard-App kann ich die Drohne beim Server registrieren und einer Organisation hinzufügen
	\item[\done] Das Onboard-App kann sich mit dem Server verbinden
	\item[\done] Das Onboard-App kann auf Wunsch die Verbindung zum Server trennen
	\item[\done] Die Drohne kann auf dem Onboard-App als inaktiv markiert werden
	\item[\done] Das Onboard-App kann sich mit der Drohne über USB Kabel verbinden
	\item[\done] Das Onboard-App kann die Verbindung mit der Drohne trennen
	\item[\done] Es kann der aktuelle Status des GPS, der Batterie und der Verbindung zum Server angezeigt werden
	\item[\done] Eine vorgeschlagene Mission kann abgelehnt werden
	\item[\done] Drone-Operator kann eine Mission annehmen und sieht die zu belandende Produkte
	\item[\done] Bevor die Drohne startet erhalte ich ein Countdown
	\item[\done] Bevor die Drohne startet erhalte ich einen akustischen Signal
	\item[\done] Während des Countdown kann der Start abgebrochen werden
	\item[\done] Die Drohne kann mit den Produkten beladen und bestätigt werden
\end{todolist}

\subsubsection{Nicht-Funktionale Anforderungen}
\begin{todolist}
	\item[\done] Die Verbindung zum Server beim Registrieren ist verschlüsselt ( Server kann nur HTTPS angesprochen werden )
	\item[\done] Die Verbindung zum Server bei der Übertragung der Drohneninformation ist verschlüsselt (überprüft mittels RabbitMQ Management Konsole)
	\item[\done] Beim Verbindungsabbruch zum Server wird die Mission weitergeführt
	\item[\done] Nach einem Verbindungsabbruch wird die Verbindung automatisch wiederhergestellt
\end{todolist}

\subsection{Customer}
\subsubsection{Funktionale Anforderungen}
\begin{todolist}
	\item Der Kunde kann sich die App aus dem Google Play Store herunterladen 
	\item[\done] Kunde kann sich die bestellbaren Produkte anschauen, ohne sich einzuloggen
	\item[\done] Kunde sieht nur nur Produkte innerhalb der Bestellzone
	\item[\done] Kunde kann eine Bestellung abschicken und sieht die voraussichtliche Lieferposition
	\item[\done] Kunde kann die Bestellung abbrechen
	\item[\done] Kunde kann sich über Google Anmelden und sieht welche Berechtigung von Google erforderlich sind
	\item[\done] Kunde kann die bestellte Ware mit PayPal bezahlen ( nur im Testmodus )
	\item[\done] Kunde kann die Bestellung bestätigen
	\item[\done] Kunde kann seine Bestellungen anschauen und dessen Status ansehen
	\item[\done] Kunde kann die einzelnen Lieferungen anschauen und dessen Status ansehen
	\item[\done] Kunde kann bei einer aktuellen Lieferung die Position der Drohne mitverfolgen
	\item Als Kunde kann ich  eine Bestellung stornieren 
\end{todolist}

\subsubsection{Nicht-Funktionale Anforderungen}
\begin{todolist}
	\item[\done] Die Verbindung zum Server ist verschlüsselt
	\item[\done] Ohne Internetverbindung wird beim Abruf der Produkte eine Fehlermeldung angzeigt
	\item[\done] Ohne GPS Verbindung wird beim Abruf der Produkte eine Fehlermeldung angzeigt
\end{todolist}




