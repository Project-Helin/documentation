\begin{longtable}{llccX}
%\begin{longtable}{\linewidth}{llccX}
\toprule
	& Name & P \footnote[1]{Probability (Eintrittswahrscheinlichkeit)} & C \footnote[2]{Cost (Kosten im Eintrittsfall)} & Beschreibung \\


\midrule
	& \textbf{NFR} &  & & \\
\midrule
R01 & JMS auf Mobilegerät & 3 \footnote[3]{Skala von 1-5, 1 sehr unwahrscheinlich bis 5 sicher} & 4 \footnote[4]{Skala von 1-5, 1 nahezu konsequenzfrei bis 5 katastrophale Folgen} & Eine JMS Verbindung auf dem Mobilgerät könnte sich als schwierig herausstellen\\
R02 & Play Framework & 2 & 1 & Das Play Framework könnte eine zeitaufwendiger Handhabung mit sich bringen.\\
R03 & Internet auf Mobilegerät & 5 & 4 & Die Internetverbindung auf dem Mobilgerät könnte instabil sein.\\
R04 & Infrastruktur Probleme & 5 & 1 & Für das Backend wird ein Server mit JMS Broker benötigt, dies könnte auf Grund von geschlossenen Ports zu Problemen führen. \\

\midrule
	& \textbf{Hardware} &  & & \\
\midrule
R05 & Kapazität der Drohne & 4 & 2 & Die Drohne kann das Gewicht des zu transportierenden Gutes nicht heben.\\
R06 & Absturz und Schäden & 5 & 5 & Die Drohne kann Abstürzen und Reperaturarbeiten können langwierig werden, auf Grund von Teilebeschaffungen\\
R07 & Postitionsungenauigkeiten & 5 & 2 & GPS könnte auf der Drohne ungenau sein.\\
R08 & Ardupilot Handhabung & 3 & 4 & Ardupilot könnte Updates während dem Betrieb verweigern. Konkret heisst das, dass während des Fluges keine neuen Positionsdaten an den Autopiloten übergeben werden können. \\
R09 & Ardupilot API & 4 & 2 & Die vorhandene Dokumentation und das API der Ardupilot Komponente könnte unausreichend sein.\\


\midrule
	& \textbf{Workflow} & & & \\
\midrule
R10 & Entwicklungsprozesse & 5 & 2 & Der Entwicklungsprozess kann sehr umständlich sein. Die Drohne muss im Gebäude programmiert werden, der Testflug aber auf offenem Gelände. Dies könnte viel Zeit in Anspruch nehmen. \\
R11 & Kommunikation im Team & 1 & 1 & Die Teamgrösse von 3 Personen könnte zu Komplikationen in der Kommunikation führen.\\


\midrule
	& \textbf{GUI} &  & & \\
\midrule
R12 & Ablademanagement & 3 & 3 & Das Ablademanagement könnte sich schwieriger Gestalten als Angenommen, es müssen viele Sicherheitsfunktionen für den Verbraucher gewährleistet werden, die schwierig zu evaluieren sind. \\
R13 & Zonedefinition & 2 & 2 & Die Gestaltung von Zonen und deren Handhabung könnte einige offene Fragen aufwerfen. \\
R14 & Höhenproblematik & 2 & 2 & Mehrere Drohnen in der selben Flugzone, folgern eine Verwendung von unterschiedlicher Flughöhe um Kollisionen zu vermeiden \\

\bottomrule
\end{longtable}