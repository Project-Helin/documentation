\setlength\LTleft{-0in}            % default: \parindent
\setlength\LTright{-0in}           % default: \fill
\begin{longtable}{llcccXccccc}
\toprule
	& {\rotatebox{90}{\textbf{Name}}} & {\rotatebox{90}{\textbf{Wahrscheinlichkeit}}} & {\rotatebox{90}{\textbf{Kosten}}} & {\rotatebox{90}{\textbf{Risiko}}} & {\rotatebox{90}{\textbf{Beschreibung}}} &
	\multicolumn{5}{c}{{\rotatebox{90}{\textbf{Stand nach Sprint}}}} \\
\midrule
	& \textbf{NFR} & & & & & 1 & 2 & 3 & 4 & 5\\
\midrule
R01 & JMS auf Mobilgerät & 3 \footnote[1]{Skala von 1-5, 1 sehr unwahrscheinlich bis 5 sicher} & 4 \footnote[2]{Skala von 1-5, 1 nahezu konsequenzfrei bis 5 katastrophale Folgen} & \orangebox & Die JMS Verbindung auf dem Mobilgerät könnte Komplikationen mit sich bringen. & \orangebox & \greenbox & \greenbox & \greenbox &  \greenbox \\
R02 & Play Framework & 2 & 1 & \greenbox & Das Play Framework könnte eine zeitaufwendigere Handhabung mit sich bringen. & \redbox & \redbox &  \greenbox &  \greenbox &  \greenbox \\
R03 & Internet auf Mobilegerät & 5 & 4 & \redbox & Die Internetverbindung auf dem Mobilgerät könnte instabil sein. & \orangebox & \orangebox & \orangebox &  \greenbox & \greenbox \\
R04 & Infrastruktur Probleme & 5 & 1 & \orangebox & Für das Backend wird ein Server mit JMS Broker benötigt, dies könnte auf Grund von geschlossenen Ports zu Problemen führen. & \orangebox &  \greenbox &  \greenbox &  \greenbox &  \greenbox \\

\midrule
	& \textbf{Hardware} & & & & & 1 & 2 & 3 & 4 & 5\\
\midrule
R05 & Kapazität der Drohne & 4 & 2 & \orangebox & Die Drohne kann das Gewicht des zu transportierenden Gutes nicht heben. & \orangebox & \greenbox & \greenbox & \greenbox & \greenbox \\
R06 & Absturz und Schäden & 5 & 5 & \redbox & Die Drohne kann Abstürzen und Reparaturarbeiten könnten lange dauern, wenn Ersatzteile nicht verfügbar sind. & \redbox & \redbox & \redbox & \redbox & \redbox \\
R07 & Postitionsungenauigkeiten & 5 & 2 & \orangebox & GPS könnte auf der Drohne ungenau sein. & \orangebox & \orangebox & \greenbox & \greenbox & \greenbox \\
R08 & Ardupilot Handhabung & 3 & 4 & \orangebox & Ardupilot könnte Updates während dem Betrieb verweigern. Konkret heisst das, dass während des Fluges keine neuen Positionsdaten an den Autopiloten übergeben werden können.  & \orangebox & \greenbox & \greenbox & \greenbox & \greenbox \\
R09 & Ardupilot API & 4 & 2 & \orangebox & Die vorhandene Dokumentation und das API der Ardupilot Komponente könnte unausreichend sein. & \orangebox & \greenbox & \greenbox & \greenbox & \greenbox \\


\midrule
	& \textbf{Workflow} & & & & & 1 & 2 & 3 & 4 & 5\\
\midrule
R10 & Entwicklungsprozesse & 5 & 2 & \orangebox & Der Entwicklungsprozess kann sehr umständlich sein. Die Drohne muss im Gebäude programmiert werden, der Testflug aber auf offenem Gelände. Dies könnte viel Zeit in Anspruch nehmen.  & \orangebox & \greenbox & \greenbox & \greenbox & \greenbox \\
R11 & Kommunikation im Team & 1 & 1 & \greenbox 
 & Die Teamgrösse von 3 Personen könnte zu Komplikationen in der Kommunikation führen. & \greenbox & \greenbox & \greenbox & \greenbox & \greenbox \\


\midrule
	& \textbf{GUI} & & & & & 1 & 2 & 3 & 4 & 5\\
\midrule
R12 & Ablademanagement & 3 & 3 & \orangebox & Das Ablademanagement könnte sich schwieriger Gestalten als angenommen, es müssen viele Sicherheitsfunktionen für den Verbraucher gewährleistet werden, die schwierig zu evaluieren sind.  & \orangebox & \orangebox & \greenbox & \greenbox & \greenbox\\

R13 & Zonedefinition & 2 & 2 & \greenbox & Die Gestaltung von Zonen und deren Handhabung könnte einige Fragen aufwerfen.  & \greenbox & \greenbox & \greenbox & \greenbox & \greenbox \\
R14 & Höhenproblematik & 2 & 2 & \greenbox & Mehrere Drohnen in der selben Flugzone, könnten zu Kollisionen führen, sofern diese auf derselben Höhe fliegen. & \greenbox & \greenbox & \greenbox & \greenbox & \greenbox\\

\bottomrule
\caption{Risiken}
\label{table:risk-table}
\end{longtable}